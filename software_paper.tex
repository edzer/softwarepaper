\documentclass{article}

\usepackage[colorlinks=true,urlcolor=blue]{hyperref}

\author{Edzer Pebesma\footnote{{\tt edzer.pebesma@uni-muenster.de}}}
\date{Version 0.1, \today}
\title{\bf The Software Paper}

\begin{document}
\maketitle

\section{Introduction}
Today, many scientific papers are based upon scientific
computing, which in turn is based on using scientific
software. Increasingly, developing original scientific
software is an integral part of the process that leads to
obtaining new scientific insights The {\em research software impact
manifesto}\footnote{\url{http://software.ac.uk/blog/2011-05-02-publish-or-be-damned-alternative-impact-manifesto-research-software}}
sais that {\em software has become the third pillar of research,
supporting theory and experiment.}

The {\em software paper} is a specific type of scientific paper
that describes an original scientific software contribution to (i)
helps other scientists reproduce previously published and carry
out new research, (ii) increase the transparency of (previous and
future) scientific research, and (iii) allow proper citing of the
software contribution when used in consecutive papers. This paper
describes the software paper, its rationale, its requirements,
and recommendations. It may yield a blueprint for software papers
published in {\em Computers \& Geosciences}.

\section{Rationale}

(How does the software paper differ from standard documentation?)

manual - tutorial - design paper

software paper describes the originality or scientific contribution,
describes the context

\section{Requirements}
\begin{description}
\item[originality] the paper should point out why the software contribution
is original, and how it improves over existing solutions: why was there
a need to develop this software.
\item[context] the paper should point how the software developed
relates or compares to existing solutions, and how existing solutions
were integrated or re-used.
\item[source code] a pointer to the source code, either a file with
a tar ball, or a repository (git, subversion).
\item [availability] the software should be made available to the 
reviewers, meaning that it is part of the paper at time of first
submission.
\item[readability] the software should should be readable, have comments
in place where relevant, and be written having in mind that others would
want to understand it.
\item[version] version number of source code, revision ID of SCM system
\item[examples] should show input, interaction (if any), and output;
the examples should be easily reproducible with the source code available.
\item[citing] the paper should cite all relevant software mentioned, 
or used to build the software described.
\item[portability] the paper should describe on which operating systems
and/or computing platforms the software runs
\item[copyright, license] the paper and software should both
clearly state where the copyrights lies, and under which license
it is distributed

\end{description}

\subsection{Requirements to reviewers}
Reviewers should primarily review the paper, but also address the
software described. Minimum checks include:
\begin{itemize}
\item can the software be downloaded?
\item can the software be installed, and run?
\item can the examples be run?
\end{itemize}
The reviewers should address the question whether the contribution
of the software paper is original, substantial, and sufficiently
embedded in the context of existing solutions and/or embedding
environments.

\section{Recommendations}
Author(s) should collect user feed-back before publication; report
on user feed-back. Discuss usability.

((See Hyndman blog))

\section{Discussion}
Persistence of web sites, and of links

\section{Links}


\begin{itemize}
\item \url{http://software.ac.uk/so-exactly-what-software-did-you-use}
\item \url{http://robjhyndman.com/hyndsight/jss-rpackages/}
\end{itemize}

\end{document}
