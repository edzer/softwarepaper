\documentclass{article}

\usepackage[colorlinks=true,urlcolor=blue]{hyperref}

% \author{Edzer Pebesma\footnote{{\tt edzer.pebesma@uni-muenster.de}}}
\date{\today}
\title{\bf Call for {\em Software Papers} in \\ Computers \& Geosciences}

\begin{document}
\maketitle

This is a call for papers for a special issue of {\em Computers
\& Geosciences} with {\em software papers}, a new paper type. The
deadline for this special issue is Jul 1, 2015. The special issue
editor is Edzer Pebesma.

With the advent of {\em computational science} as a substantial
part of many natural sciences including the geosciences, the
development of scientific software to implement, apply and evaluate
computational models has strongly increased. Analogue to sharing
scientific findings, there is an increasing need to publish this
software, and communicate this through scientific papers. Although
many papers in {\em Computers \& Geosciences} come with software
implementation, and the {\em software review} exists as a special
paper type, the journal so far had no {\em software paper} type
where software is described as the main contribution.

A {\em software paper} presents a piece of software, usually
developed by the authors, to a scientific audience.  It describes
which scientific problems the software solves, which computational
models it implements, why there was a need for this software,
how it relates to other solutions, under which license it is
being made available, where it can be found, and demonstrates
how the software works by one or more real examples. All this is
done with complete references to the relevant sources. Finally,
it discusses implementation details (e.g. the architecture chosen,
user interface), usability, portability, and limitations.  As any
scientific paper it should be written concisely and factual, and
should try to convince its readership about the scientific progress
made by publishing this software\footnote{In contrast, a user manual
is part of the software and describes the complete functionality, may
be written less formally, and may omit comparisons, implementation
details, and discussion.}.

Software papers are submitted along with the software they
describe, either as files or as a reference to a web site or a
github repository.  Transparency of scientific conduct requires that
the software is published under a useful open source licence. The
submission should include everything needed to replicate the
example(s) presented in the paper. The review process includes
the act of verifying that the software works, and reproduction of
the examples.

\end{document}
