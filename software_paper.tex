\documentclass{article}

\usepackage[colorlinks=true,urlcolor=blue]{hyperref}

\author{Edzer Pebesma\footnote{{\tt edzer.pebesma@uni-muenster.de}}}
\date{Version 0.1, \today}
\title{\bf Software Papers in \\ Computers \& Geosciences}

\begin{document}
\maketitle

\section{Introduction}
Today, many scientific papers are based upon scientific
computing, which in turn is based on using scientific
software. Increasingly, developing original scientific
software is an integral part of the process that leads to
obtaining new scientific insights. The {\em research software impact
manifesto}\footnote{\url{http://software.ac.uk/blog/2011-05-02-publish-or-be-damned-alternative-impact-manifesto-research-software}}
for instance argues that {\em software has become the third pillar
of research, supporting theory and experiment.}

We describe the {\em software paper} as a specific type of scientific
paper that describes an original scientific software contribution
which has the goals to (i) help other scientists understand
and reproduce previously published research, (ii) increase the
transparency of the scientific research where the software was used,
(iii) allow reuse of the software to carry out new research, and
(iv) allow proper citing of the software contribution, e.g. when
it is used in consecutive papers. Following goal (ii), we will
only consider papers describing open source software.  We describe
the {\em software paper}, its purpose, its requirements, and
recommendations. We aim at providing a blueprint for future software
papers published in {\em Computers \& Geosciences}, or elsewhere. It
has consequences to the review process, as reviewers should address
the software published too.

\section{Purpose}

A {\em software paper}, as described here, needs to be an original
scientific contribution. Where usual scientific paper describe
novel theories or findings that follow from analysing data, the
software paper describes original scientific software. As with all
scientific papers, the paper should describe the software in the
context of previous work. It should argue why there was a need
for the software, describe similar or competing solutions, and
demonstrate the use of the software. It should properly reference
the relevant related literature, be it other software papers or
papers where the software described was put into practice. It should
also discuss the virtues of the software, and describe its use,
and the user experiences.

% manual - tutorial - design paper
% software paper describes the originality or scientific contribution,
% describes the context

\section{Requirements}
The following minimum requirements need to be maintained in order
to obtain comprehensive description of the software publishe.
\begin{description}
\item[originality] The paper should point out why the software
contribution is original, and how it improves over existing
solutions. It should answer the question why there was a 
need to develop the software.
\item[context] The paper should point how the software developed
relates or compares to existing solutions, and how existing solutions
were integrated or re-used.
\item[source code] A pointer should be given to the source code, 
which can either be a URL pointing to a tar ball, 
more typicall a source code repository (git, subversion). 
\item[versioning] The paper
should clearly mention which version of the software is described,
either encoded in the file name, or as the tag, revision, or commit
number identifier in a repository.
\item [availability] The software should be available to the 
reviewers at the time of submitting the paper, in source code form, 
meaning that it is part of the paper at time of first submission, 
and subject to review.
\item[readability] The software should should be readable, have comments
in place where relevant, and be written having in mind that others would
want to understand it. The software should be organized in a way that makes
it useful for prospective users.
\item [portability] The paper should describe the platform(s) targeted
by the software, and describe {\em requirements} for installing, compiling (if
needed) and running the software. Complete {\em instructions} for installing
the software are not needed in the paper, but should be easily identifiable
(e.g. in README files) in the source code itself.
\item[examples] The paper should demonstrate show how the software works
with real examples, describing input, interaction (if any), and output.
The examples should be easily reproducible with the source code available.
\item[citing] The paper should cite all relevant software mentioned, 
or used to build the software described. This does not include generic
tools (like {\tt make}, or {\tt libc}) used for most software, but specific
tools on which the described software critically depends.
\item[copyright, license] The paper and software should both
clearly state who owns the copyrights, and under which license
it is distributed.
\end{description}

\subsection{Requirements to reviewers}
Reviewers should primarily review the paper, but also address the
software described. Minimum checks include:
\begin{itemize}
\item can the software be downloaded?
\item can the software be installed, and run?
\item does it run the examples shown in the paper?
\end{itemize}
The reviewers should address the question whether the contribution
of the software paper is original, substantial, and sufficiently
embedded and/or discussed in the context of existing solutions
and/or embedding environments.

\section{Discussion}
Research software is rarely written with the main purpose of writing
a paper about it, and publishing this. Doing this nevertheless makes
only sense if the author believes that a certain readership is helped
with this, and will be inclined to read and hopefully cite the paper.
Citations may be seen as a reward that users are typically inclined
(if not morally obliged) to give when they use the software described
for their scientific work. Although direct pointers to the software
itself, e.g. by URLs, tags like CRAN:packagename\footnote{TODO},
or DOIs pointing to the software\footnote{DOI-from-github-TODO}
in the paper, from a software paper the readership may expect
that the software has been properly described and discussed for a
scientific audience.

In general, it is recommended that authors write a software paper
after a certain amount of user feed-back has been collected,
and after the software has reached a certain stability. This
allows to report on usage, user feedback, and reflect on
its usability\footnote{HyndManBlogJSS}. Needless to say, publishing
software requires properly paying attention to software testing,
naming conventions, organisation, complexity, scalability,
intelligable comments and user documentation, and so on.

It is not unusual that research software is being further developed
after a paper on it has been published; the paper is usually does
not allow updating. It is therefore useful that a fixed version
that matches the published paper is identifiable; this can be done
by a source code git clone made by the journal editor (Computers
\& Geosciences), or by a full copy of the software (Journal of
Statistical Software). Alternatively, an updated copy of the paper
can be provided by the software author, along with the software, as
is often done with vignettes accompanying R packages\cite{xx}.

Persistence of web sites, and of links

long-term curation and github.

[pointers to other descriptions of / requirements to software papers]



\section{Links}

\begin{itemize}
\item \url{http://software.ac.uk/so-exactly-what-software-did-you-use}
\item \url{http://robjhyndman.com/hyndsight/jss-rpackages/}
\end{itemize}

\end{document}
